\documentclass[a4paper, titlepage]{article}
\title{Sobre el LSD.}
\author{Erik López de la Fuente \\ Ana M. Juárez Guerra \\ IES José Hierro (Getafe)}
\date{March 19, 2024}

\usepackage[utf8]{inputenc}
\usepackage[spanish]{babel}
\usepackage[table]{xcolor}
\usepackage[autostyle]{csquotes}
\usepackage{enumerate}
\usepackage{url}
\usepackage{caption}
\usepackage{amsmath}
\usepackage{amsfonts}
\usepackage{graphicx}
\usepackage{subcaption}
\usepackage{float}
\usepackage{hyperref}

\begin{document}

\maketitle

\begin{abstract}
En este documento estudiamos la historia y farmacología de la dietilamida del ácido lisérgico (LSD), desde la primera mención escrita del ergot en Europa (y sus repercusiones como remedio herbal y causante del ergotismo) hasta la síntesis en 1938 por Albert Hofmann. Introducimos conceptos básicos de la neurociencia para entender la acción de esta molécula sobre el cuerpo y después explicamos cómo se manifiestan sus efectos sobre el individuo, analizándolo desde la perspectiva de la psicología. Finalmente contamos sus usos y abusos subsecuentes en el ámbito de la terapia, la recreación y el arte y vemos su influencia sobre los movimientos sociales de la segunda mitad del siglo XX.

Palabras clave: LSD, drogas psicodélicas, terapia psicodélica, ergot, movimiento hippie
\end{abstract}

\section*{}
\thispagestyle{empty}

Agradecimientos tanto al equipo directivo del IES José Hierro como a mi tutora de proyecto Ana M. por la paciencia administrativa que he conllevado.

Abreviaturas usadas: LSD (d-LSD, la dietilamida del ácido lisérgico), SNC (sistema nervioso central), SNP (sistema nervioso periférico), GABA (ácido $\gamma$-aminobutírico), 5-HT (5-hidroxitriptamina, sinónimo de serotonina), 5-HT$_{\textrm{XY}}$ (receptor de serotonina Y de la clase X), DMT (dimetiltriptamina), VTA (área tegmental ventral), NAc (núcleo accumbens), [Na$^+$] y [K$^+$] (concentraciones de iones de sodio y potasio), SSRI (inhibidor selectivo de recaptación de serotonina), TDAH (trastorno por déficit de atención e hiperactividad), CIA (agencia central de inteligencia), $\mu$g (microgramos).

\clearpage


\tableofcontents

\newpage

\section{Introduction.}

In winter harvests, purple-colored stains show up in rye flowers. These stains, initially thought to be excess sap, encapsulate the monster responsible for two of medieval Europe's endemic diseases. They hold chemical compounds so varied that when they don't cause spasms, they serve as medicine, and in a bizarre coincidence they gave us the psychedelic drug that marked the 20th century, from the Vietnam War to hippie pacifism.

Lysergic acid's diethylamide or LSD, born from this havoc, is a molecule with miraculous properties both pharmacologically and psychiatrically. With a huge potenci in microgram dosages, it has starred the most recent social movements. What therapeutic potential does this substance have? What were the consequences of its discovery? We present a study on its natur and history, the effects it causes on the individual and the associated risks, and over all, the biological fundamentals which explain its action upon conscience.

\newpage

\section{Ergot's history.}

The presence of rye stains in somewhere's whereabouts foreshadowed one out of two terrible diseases --- almost endemic to some regions. West of the Rhine, patients were bothered by sensations in a limb which, in a few weeks, turned into burning so intense that it was called \enquote{Saint Anthony's fire}. Luckily for the unfortunate, the pain ceased just to be substituted by numbness. The now cold and pale limb blackened, shrank and mummified. The gangrene expanded until finishing with a premature death. On the other side there wasn't much luck either. The sick were attacked by frequent seizures: first fingers, then limbs, hips and finally, the whole body. Covered in vomit they howled in pain, and when it got to the head, the epileptic episodes began as a prelude to blindness, deafness, and death.

Between the 17th and the 18th century, it was discovered that this purple badges were actually the mycelium of a fungus known now as \textit{ergot} (species belonging to the \textit{Claviceps} genus). Furthermore, it was confirmed that the substances this fungus produced directly caused those diseases, and so the formers were named \textit{ergotism}. Ergot grows parasitically in rye flowers and falls producing fruiting bodies that release ascospores, infecting new grain and repeating the cycle.

Rye isn't new and neither is ergot, and even then we hold no written mention of it in Europe until a book of herbal remedies from the 16th century, in which its obstetric use (to precipitate labor) is described (Figure \ref{ergotbook}). Almost no other book at the time mentions ergot, so either botanists didn't know about it or didn't consider it relevalt. The link between ergot and ergotism generated a stigma against its use as a treatment. Despite having been used by generations of midwives previously, it was forbidden in many european regions. In the New World however, its use was popularised in a boiled powder format (\textit{Pulvis parturiens} or \textit{Pulvis ad partum}) thanks to doctors John Stearns and Akerly. Sold in pharmacies, prescribed by physicians and used by midwives, a cultural transfer to a land free of the prejudices europeans had acquired through centuries of experience. In 1813, thanks to a very successful dissertation from Oliver Prescott (which was translated and exported to Europe), a general great interest on the medicinal properties of this medicine arises back. In the 20th century, ergot was produced in tons anually in cities like Vigo, Lisboa and Leningrad. 

\begin{figure}[H]
	\centering

	\includegraphics[height=.3\textheight]{media/1-ergotcover.png}
	\includegraphics[height=.3\textheight]{media/1-ergotparagraph.png}
	\caption{One of the first written mentions of ergot where its obstetric use is described. The full story of the fungus is exposed to an exquisite detail in the book \textit{Ergot and ergotism}, from which this fragment is taken.}
	\label{ergotbook}
\end{figure}

\subsection{Albert Hofmann's investigations.}

Having come out of the University of Zürich in 1929, chemistry student Albert Hofmann started working in the laboratories of a company named Sandoz. In previous decades, Sandoz had been investigating ergot's alkaloids: substances that, they thought, produced in organisms both the therapeutic and the toxic effects of ergot. If it was achieved to separate the components responsible for uterine contractions from those responsible for ergotism, valuable medicines could be obtained. These studies were leaded by Arthur Stoll, and Albert Hofmann would carry them under his supervision. 

Hofmann made quick progress, isolating the uterotonic and haemostatic components, very useful during labor and which made ergot such a popular remedy in past times; and the common nexus of many ergotic alkaloids: lysergic acid (not to be confused with its diethylamide, LSD. Lysergic acid on its own isn't hallucinogenic) (Figure \ref{alkaloids}).

\begin{figure}[H]
	\centering
	\includegraphics[width=\linewidth]{media/2-alkaloids.png}
	\caption{Many of the ergot alkaloids are derived from lysergic acid (shown in black) through an amide bond to other compound (shown in red). LSD (bottom right) is lysergic acid's di-ethyl-amide, that is, lysergic acid joined to two ethyl groups through an amide bond.}
	\label{alkaloids}
\end{figure}

Using lysergic acid, Albert Hofmann synthesized dozens of compounds, the twenty-fifth of which was its diethylamide: LSD-25 or simply LSD (\textit{\textbf{L}yserg\textbf{s}äure-\textbf{D}iethylamid}). At first it was considered irrelevant and it wouldn't see light again until 5 years later, when Hofmann, intrigued by \enquote{the feeling that this substance could possess properties other than those established in the first investigations}, decided to synthesize it again. During the process he suffered a series of strange sensations that he would then report to Stoll:

% Cita
\let\oldquote\quote
\let\endoldquote\endquote
\renewenvironment{quote}[2][]
  {\if\relax\detokenize{#1}\relax
     \def\quoteauthor{#2}%
   \else
     \def\quoteauthor{#2~---~#1}%
   \fi
   \oldquote}
  {\par\nobreak\smallskip\hfill(\quoteauthor)%
   \endoldquote\addvspace{\bigskipamount}}

\begin{quote}[LSD: my problem child]{Albert Hofmann}
	\enquote{Last Friday, April 16,1943, I was forced to interrupt my work in the laboratory in the middle of the afternoon and proceed home, being affected by a remarkable restlessness, combined with a slight dizziness. At home I lay down and sank into a not unpleasant intoxicated-like condition, characterized by an extremely stimulated imagination. In a dreamlike state, with eyes closed (I found the daylight to be unpleasantly glaring), I perceived an uninterrupted stream of fantastic pictures, extraordinary shapes with intense, kaleidoscopic play of colors. After some two hours this condition faded away.}
\end{quote}

He theorized that, in an act of carelessness, a bit of LSD-25 had come into contact with his skin, causing these effects. Were this to be true, the effective dose of this substance would have to be very small. To confirm it, he decided three days later to consume 0.25 miligrams, an amount he considered wise. He felt as if a devil was geting ahold of his mind, body and soul. He feared to be in a transition to death. He had actually taken two and a half times the dose we now consider standard. Despite that, his physical state was correct (apart from his dilated pupils), so with the pass of time and anxiety, he could enjoy the fantastic images that flashed before him. For his surprise, he didn't feel any type of hangover the next day, but a sensation of wellbeing that bloomed in him. He saw the world as newly made. LSD produces hallucinogenic effects similar to those of mescaline or psylocibin (alkaloids from the Peyotl cactus and psylocibe mushrooms respectively) in doses dozens of times inferior and without chronic toxic effects (at those doses). The main danger --- according to Hofmann --- was what could be done or felt during the state of drunkenness. To understand the details of LSD's operation, it's necessary to know in depth what it acts upon.

\newpage

\section{The basics of the nervous system.}

The nervous system is the most complex chemical machine known and to be known. Every attempt to study it is condemned to constant theoretical gaps and wide borders, most of them in perpetual darkness. This condition arises from an organizational complexity and not a fundamental one, as the individual elements that comprise it are mostly well understood. Ignoring neuroscience's limitations as a novel field, the knowledge of perception is limited by metaphysical matters. Even after the reader has accepted his own existance, some dilemmas are put into place: there are cases of patients who have suffered cerebral injuries and show changes in behavior such that they become unrecognisable to their loved ones. Trying to explain such phenomena using high abstraction concepts like \enquote{personality} or \enquote{soul} is interesting as intellectual foreplay, but is also a well of contradictory answers. For that reason, when studying these phenomena we can't but limit ourselves to finding patterns in that which we can observe and write a great \enquote{WE DON'T KNOW} in what surrounds it.  

What comes now isn't going to be simple, but in essence it summarizes into:

\begin{enumerate}
	\item The nervous system is made up of cells called neurons.
	\item A neuron can communicate with others through one-directional signals.
	\item Chains and webs of neurons compose functional circuits that can configure simple behaviors (like moving a leg by reflex) up to very complex ones (like emotional state).
\end{enumerate}

Altering the behavior of groups of neurons in very subtle ways can cause chain reactions noticeable at larger scales. This is what happens after administering a drug, and will be seen in detail proximately.

\subsection{The building blocks of the nervous system.}

Entering his thirties and in search for economic stability, Camilo Golgi started to work in a hospital for chronic patients in Abbiategrasso. In his drawer's kitchen, with no more than a microscope and self payed utensils, he built a laboratory to which he came back after taking care of his patients to investigate about a mystery that chased him: the entity responsible for nervous communication. Golgi had discovered a method with which he managed to dye some nervous system's cells in a black color. This histological discovery played with the posibility of understanding the mind, something inimaginable in the 19th century. Ironically, he became eclipsed by the Spanish scientist Santiago Ramón y Cajal who, using his method, elaborated theories much better than his own.

These cells --- called nerve cells or neurons --- had a branching interconnected structure which formed wide networks. Both scientists disagreed on a simple detail: were neurons independent cells as Cajal thought or did they act as a continuous tissue without internal separation as Golgi thought? This confrontation was closed definitely with the invention of electron microscopy and the discovery of a small gap between neurons: the synaptic space (Figure \ref{synapse}-C$_1$).

\begin{figure}[H]
	\centering

	\includegraphics[width=\linewidth]{media/3-synapse.png}
	\caption{(C$_1$) Cells from the nervous system seen under an electron microscope. In orange, neurons separated by the synaptic gap. In blue, an astrocyte that supplies them. (C$_2$) Schematic view. Upon an action potential, calcium channels open. Calcium induces the fusion of synaptic vesicles with the cell membrane (exocitosis), freeing neurotransmitters (glutamate) into the synaptic space. Neurotransmitters join to receptors on the postsynaptic neuron, causing many possible effects. Then they are taken back by the neuron or by the astrocyte.}
	\label{synapse}
\end{figure}

This conflict was however the product of a war that was being fought since Descartes' times. The war between those who thought of the nervous system as a whole and those who thought it was divided into functional, localized parts. Sideproducts of this fight were people like Franz Joseph Gall, who using his theory for the localization of parts, ideated phrenology: a method to determine personality through the measurement of the head's shape which was later used by racism theorists as a scientific justification (Figure \ref{gall}). This idea was opposed to opinions like those from Pierre Fluorens who, testing out Gall's thesis, concluded that all of the brain's parts took care of all mental operations. A reaction not only scientific in nature but cultural, as the reduction of the soul to the internal activity of a network of organs was something unacceptable for the european context.

\begin{figure}[H]
	\centering

	\includegraphics[width=\linewidth]{media/4-gall.png}
	\caption{Functional map of the head according to Franz Joseph Gall. His methods to associate regions and functions were quite poor and his theories were proven wrong, but the idea of using craneum measurements as personality indicators was used as a base for racist theories.}
	\label{gall}
\end{figure}

Current thinking of this is somewhat mixed. While it is recognized that very complex tasks like emotions are performed by many distributed circuits on varied regions, it is also accepted that these can be broken down into much simpler tasks which are generally localizable.

\subsection{Structure and organization.}

The environment provides organisms with a continuous stream of information: electromagnetic radiation, variations in air pressure, vibrations on the ground... There exist organs with receptors capable of reacting to these variations in the environment: the sensory organs. These variations perceived by an organism through these organs are named \textit{stimuli} (note that depending on the level of abstraction, a stimulus can be anything from the ring of a bell to the impact of a single photon). When a sensory organ receives a stimulus, it produces a signal on the nervous system which generally ends up with the activation of another organ which we name \textit{effector}. The action that gets carried out is named \textit{response}, and if the response is complex enough, we call it \textit{behavior}.

Between the stimulus and the response lies the nervous system: a set of cells which conduct electrical signals through different parts of the organism. It is divided in \textit{central nervous system} (CNS), made up of the encephalon and the spinal cord, and \textit{peripheral nervous system} (PNS), made up of all the nerves that branch from the latter. The PNS acts as a highway of information between the CNS and the effector and sensory organs, whereas the CNS takes charge of integrating and processing the contributions of the PNS to offer an adequate response.

For example, upon the presence of a tarantula in one's leg (stimulus), sensory receptors in the skin will send signals communicating the information to the CNS via chains of neurons (nerves) from the PNS. After looking at the tarantula, the CNS, configurated throughout years by past experiences with arachnids, cultural notions on tarantulas or by mere genetic predisposal may activate through the PNS muscles from the leg to shake it and glands to secrete stress and fear hormones. The PNS has then two ways: an \textit{afferent} one (from the sensory organs to the CNS) and an \textit{efferent} one (from the CNS to the effector organs).

Not all responses require the encephalon. Many reflex behaviors are self-contained in the spinal cord's circuitry . However, the encephalon offers a very useful capacity for interpretation in more complex living things.

The cells from the nervous system which send and receive signals are the neurons (Figure \ref{neuron}), but in the nervous system there are also many other cells like glia, which serve as support for neurons (Figure \ref{glia}). From microglia, with immunitary function, to macroglia, which covers neurons with a lipidic layer of mielin, nourishes them by controlling the pass of substances from the bloodstream and controls the concentrations of ions, neurotransmitters and exogenous elements (Figure \ref{synapse}).

\begin{figure}[H]
	\centering
	\includegraphics[width=\linewidth]{media/5-neuron.jpg}
	\caption{The neuron.}
	\label{neuron}
\end{figure}

\begin{figure}[H]
	\centering

	\includegraphics[width=\linewidth]{media/5-glia.png}
	\caption{Macroglial cells.}
	\label{glia}
\end{figure}

\subsection{The neuron.}

Every neuron is composed of three parts: a soma or cell body, where the metabolic processes common to every cell happen; the dendrites, a branched structure through which signals enter; and the axon, also branched and which sends signals to other neurons. The axon carries out an integrating function, as based on the many received signals it \enquote{decides} whether or not to fire a signal to other neurons. The neuron that sends a signal is called \textit{presynaptic neuron}, and the one that receives it is called \textit{postsynaptic neuron}. Obviously these are relative terms, as every neuron is both presynaptic and postsynaptic. How is the very fast communication observed in the nervous system possible? A molecular analysis is required.

At rest state the neuron is polarized, that is, just as a battery, it constantly mantains a difference in electric potential between its interior and its exterior of about 70 milivolts less. We say the interior is at -70 mV with respect to the exterior. Where does this potential come from and why isn't there an instant short circuit which calms it? It all depends on a delicate equilibrium (Figure \ref{action}).

As every cell is, the neuron is also delimited by a phospholipid bilayer, and hence it doesn't allow the transit of charged substances like ions. Because ions can't pass through, the flow of electricity is impossible, and so the aforementioned short circuit never happens. This bilayer however is interrupted by transmembrane proteins called \textit{channels}, whose structure form a \enquote{tunnel} through which ions can flow. These channels are capable of discriminating ions, allowing some to pass but not others. In addition, they can open and close under different conditions, allowing or not allowing flux.

Outside the cell, the concentration of sodium ions (Na$^+$) is high, whereas inside is low. On the other hand, the concentration of potassium ions (K$^+$) outside is low, whereas inside is high. This difference in concentration is maintained by the kidneys and the sodium-potassium pump, a protein which takes into the cell two K$^+$ ions for each three Na$^+$ ions it takes out, consuming energy. Ions are then exposed to an electrochemical force: the sum of the force that tries to even the potential on the outside and the inside plus the diffusion force that tries to even the inner and outer concentrations. These forces produce a sway of ions which develops until reaching equilibrium. The internal potential during this equilibrium is called \textit{membrane potential}.

At rest state, open channels are more permeable to K$^+$ than to Na$^+$, so, counting the action of the Na$^+$-K$^+$ pump, the equilibrium potential is at -70 mV. However, certain phisical-chemical modifications can alter this. For example, a considerable entrance of Na$^+$ can get the interior of the cell up to -55 mV (a potential known as \textit{threshold}). In these conditions, ion Na$^+$ channels that are activated by voltage open, destabilizing the system and causing a massive influx of sodium. The new equilibrium potential in these conditions is at around +55 mV, and is known as \textit{action potential}. The closing of these channels and the subsequent action of the Na$^+$-K$^+$ pump then recovers the initial state of the system until the next event.

\begin{figure}[H]
	\centering
	\includegraphics[width=\linewidth]{media/6-potencial.png}
	\caption{Transverse view of the neuron's membrane. IT contains potassium and sodium channels (A), which open mainly during the action potential. The Na$^+$-K$^+$ pump (B) maintains the inner concentrations of these ions.}
	\label{action}
\end{figure}

\begin{figure}[H]
	\centering
	\includegraphics[width=\linewidth]{media/6-potencialgraph.png}
	\caption{Membrane potential through time.}
	\label{actiongraph}
\end{figure}

\subsection{Neurotransmission.}

Ion channels can open and close due to the joining of certain molecules we call \textit{neurotransmitters}. Each receptor corresponds to a single neurotransmitter. Neurotransmitters are released by the presynaptic neuron, which stores them in small vesicles. During the action potential, Ca$^{2+}$ channels open. The presence of calcium causes reactions over the vesicles' proteins and the cell membrane which produces the fusion of both (exocytosis). This fusion, which can be full or partial, frees up neurotransmitters contained in the vesicle into the synaptic space, where they travel until reaching receptors in the postsynaptic neuron, to which they join. Molecules that join to receptors are called \textit{ligands} (Figure \ref{synapse}).

When joining to receptors they can make it easier or harder for the channels to open, whether that be directly (on ionotropic receptors) or indirectly by producing metabolic changes in the cell (in metabotropic receptors). While it's true that the performed action depends on the receptor and less so on the transmitter, many neurotransmitters tend to be exclusively excitatory (they increase the chance for the channel to open, thereby increasing the chance that a signal is produced) or exclusively inhibitory (they decrease the chance). For example, aminoacids glutamate and GABA are the main components in charge of excitatory and inhibitory transmission respectively.

The most important neurotransmitters for our case are catecholamines (dopamine, norepinephrine and epinephrine), in charge of excitatory functions, fear, rage and reward; and serotonin: crucial in the regulation of mood, perception and dreams and very related to LSD.

\subsection{Example: the patellar reflex.}

The neural circuit that causes the knee-jerk reflex is a perfect example of what we've exposed so-far. With the leg flexed, the knee's tendon is striked. Its extension causes stress on the quadriceps, which causes an activation of sensory neurons, producing an excitatory afferent nervous impulse to the spinal cord. In the spinal cord, two efferent neurons are activated: an excitatory neuron to the quadriceps which releases acetylcholine on the muscle, a neurotransmitter responsible for its contraction; and an inhibitory neuron which avoids the activation of the analogous neuron in the complementary muscle: the hamstring (Figure \ref{sn}). This way, the quadriceps' contraction is coordinated with the hamstring's non-contraction (and hence extension), resulting in the leg extending. Excitatory neurons communicate using glutamate as neurotransmitter, inhibitory ones: GABA or glycine. The contraction of the muscle is mediated by the liberation of acetylcholine upon it.

\begin{figure}[H]
	\centering
	\includegraphics[width=.8\linewidth]{media/7-sn.png}
	\caption{The patellar reflex.}
	\label{sn}
\end{figure}

Like bees in a hive, the neurons get grouped in organizational levels more and more complex, producing faculties like senses and memory, building up the emergent phenomena that are the individual and --- maybe --- conscience.

In these more complex tasks we don't tend to bother with individual nerves and neurons, but rather observe how populations of neurons are activated in approximate regions on the encephalon. These regions can interact with each other. Sometimes they are associated anatomically, others functionally, and others by the neurotransmitter they use to communicate. That way, when someone is happy, there may be more activity in this or that region, whereas when they're scared, maybe the regions that secrete this or that neurotransmitter are inhibited.

Unfortunately, a \enquote{leap of faith} needs to be taken to establish a connection between psychology and neurobiology. This makes it very difficult to treat disorders related to the mind. Because the biological fundamentals behind them aren't fully understood, it isn't possbile to administer an infallible drug to cure them. This knowledge gap allows for the intrusion of shamanism and vedisms, although usually detached from their ritualistic-spiritual component in modern medicine.

\newpage

\section{El LSD sobre el cerebro.}

La introducción de entidades exógenas como el LSD u otras sustancias al organismo produce un desequilibrio con toda clase de consecuencias.

\subsection{¿Qué es una droga?}

Los organismos consumen diversas sustancias del entorno. Algunas son asimiladas inmediatamente y convertidas en materia y energía o producen excrementos. Llamamos a estas alimentos. Otras no lo son tan fácilmente: por condiciones genéticas, el cobre puede acumularse en el hígado con consecuencias graves (enfermedad de Wilson). Pero otras sustancias, en lugar de acumularse, desencadenan reacciones inmensas sobre el organismo. Estas son de especial importancia en medicina, pues al ser similares en estructura y/o función a sustancias endógenas al organismo, pueden ser utilizadas para regularlo directa o indirectamente.

Muchas sustancias pueden servir entonces como tratamiento para distintas enfermedades (como ocurre con la insulina para la diabetes). Pero por su misma naturaleza, su presencia en niveles altos en el cuerpo puede también resultar tóxica: la aspirina por ejemplo es letal en cantidades de unos 250 miligramos por kg de peso corporal (mg / kg). Por supuesto lo letal no es la aspirina en sí, sino la dosis respecto de la medida del peso corporal. Es esta doble acción de remedio y veneno la que describe la palabra griega antes mencionada \enquote{phármakon}. Era un hecho incluso entonces que la distinción entre remedio y veneno no depende de la sustancia, sino de la dosis.

Entre los no-alimentos encontramos sustancias que no solo actúan de manera somática (como el cianuro de potasio o el acetaminofén), sino que también inciden sobre el sistema nervioso, afectando a la percepción y la emoción (entendida como las reacciones psicofisiológicas internas del individuo ante estimulos importantes). A estas sustancias las llamamos \enquote{drogas}.

Es importante notar que el término \enquote{droga} es polisémico, y a veces se usa para denominar exclusivamente a sustancias con efecto sobre el SN, mientras que otras se emplea para referirse a cualquier cuerpo químico utilizado en medicina (la palabra \enquote{drug} recibe este uso particularmente en la lengua inglesa).

Por último es importante reconocer que muchas drogas son ilegales y, como tal, el término guarda otro significado en el ámbito del derecho. La relación de la humanidad con las drogas siempre ha sido conflictiva, y de manera cultural o sistemática se ha perseguido el uso de unas u otras. En torno al último tercio del siglo XX se establece una normativa más estricta en países como Estados Unidos con el Controlled Substances Act y \enquote{droga} pasa a designar a cualquier sustancia así designada por este estatuto.

Para evitar ambigüedades, en este documento emplearemos las siguientes definiciones:

\begin{itemize}
	\item Fármaco: cualquier sustancia utilizada como tratamiento para una enfermedad.
	\item Droga: cualquier sustancia que interactúa con el sistema nervioso produciendo cambios en la percepción y la emoción.
\end{itemize}

\subsection{Terminología}

Existe un conjunto de términos asociados tanto a drogas como a fármacos que definimos cuantitativa y cualitativamente. En primer lugar tenemos aquéllos referentes a la dosis. Distintos organismos difieren en sus características particulares y no reaccionarán de la misma manera a dosis iguales, es así que nuestras afirmaciones serán estadísticas y no particulares.

Es común administrar una dosis determinada a una población y observar sus efectos. La dosis efectiva media (ED50) es aquélla que produce los efectos deseados sobre el 50\% de los individuos a los que se da (puede haber distintas dosis efectivas dependiendo de los efectos que se desee producir). La dosis letal media (LD50) es aquélla suficiente para acabar con la vida del mismo porcentaje de población. De estas dos se extrae el margen de seguridad, que es la proporción entre dósis activa media y dósis letal media.

Tanto fármacos como drogas pierden parcial o totalmente su efecto con el uso continuado, este fenómenos es conocido como tolerancia. Repetidas dosis diarias de aspirina generan una tolerancia que la hace mucho menos efectiva. Esta capacidad de adaptación del cuerpo no es igual para todas las sustancias, por lo que la tolerancia que induce cada una ha de ser estudiada individualmente.

En la literatura relacionada a las drogas se suele usar la tolerancia como medida sinónima de propensión al abuso. Si bien razonable, se puede matizar. Ciertamente una droga como la heroína generará una rápida adaptación, por lo que el usuario, para sentir los mismos efectos, habrá de dejar un margen entre uso y uso o, como es común, aumentar la dosis. La adaptación del organismo al fármaco sin embargo no es pareja, y un individuo puede dejar de percibir los efectos relajantes de un opioide pero seguir sufriendo la misma depresión respiratoria. Un aumento de la dosis en este caso plantea el riesgo de intoxicación aguda. Incluso si se consigue una adaptación pareja, el uso continuado de la droga termina generando una intoxicación crónica, como es habitual en casos de alcoholismo. Un menor grado de adaptación implica límites de toxicidad más rígidos, reduciendo el riesgo de intoxicación crónica pero haciendo más fácil alcanzar dosis perjudiciales o letales.

Que el cuerpo se adapte a fármacos supone cambios metabólicos internos que hacen que el organismo necesite la presencia de la sustancia para mantener su equilibrio. Se dice entonces que el cuerpo ha generado dependencia a la sustancia. cuando la sustancia es eliminada - por ejemplo por interrumpirse su uso continuado - se produce un fenómeno de desestabilización que se manifiesta en reacciones mensurables conocido como síndrome de abstinencia. Uno empieza a vislumbrar aquí los mecanismos biológicos que conducen a la adicción, esto es, la búsqueda compulsiva de nuevas dosis, sin embargo hemos de hacer un comentario adicional al respecto.

\subsection{Adicción y dependencia.}

Mismos estimulos provocan distintas respuestas dependiendo de variables internas como el hambre o el estado anímico. Ante la presencia de agua y alimento, el depredador prioriza aquélla necesidad más urgente en ese momento. A ese conjunto de variables internas y su influencia sobre las externas lo llamamos \enquote{motivación}. La mayoría de las motivaciones buscan saciar un desequilibrio interno a corto plazo como la sed, pero algunas expanden su vista a imperativos biológicos como la perpetuación de la especie.

Cuando un agente estimula los sistemas de recompensa del cerebro de manera artificial – llegando a predominar sobre recompensas naturales, incluso en detrimento físico o psicológico del individuo – se dice que existe una adicción. La adicción se define como un síndrome crónico caracterizado por la búsqueda compulsiva de cierta droga, incluso ante consecuencias negativas físicas y personales.

Este fenómeno es estudiado principalmente en animales. Por ejemplo, cuando una rata es encerrada y provista de un mecanismo con el que estimular eléctricamente ciertas regiones de su cerebro (como una palanca), la rata tirará de la palanca incluso en condiciones de inanición, llegando a ignorar la comida que se le ofrezca. La persecución de una mete artificial en perjuicio de la necesidad biológica forma un paralelo interesantísimo con la adicción a drogas. Comprender las vicisitudes de la adicción implica entender los sistemas de recompensa del cerebro, algo que aún se está investigando.

Al analizar la actividad cerebral que suscitan distintos comportamientos hedónicos observamos que ciertas regiones muy concretas se activan de manera consistente. Una de estas regiones es el área tegmental ventral (VTA por sus siglas en inglés), una región del cerebro con muchas proyecciones al núcleo accumbens (NAc). La comunicación entre estas dos zonas se realiza por medio de un neurotransmisor llamado \enquote{dopamina}, y las neuronas que la utilizan son \enquote{dopaminérgicas} (Figura 8).

Algunas drogas como la cocaína o las anfetaminas son capaces de fortalecer la transmisión dopaminérgica, incrementando la actividad de los sistemas de recompensa como el de la VTA. Estas drogas no solo producen adicción física por su síndrome de abstinencia, sino que debido a esta perturbación neuronal, causan también adicción psicológica.

% Figura

Esta explicación de la adicción sin embargo sigue sin ser completamente sólida. Identificar la dopamina con un \enquote{neurotransmisor del placer} es incorrecto. En ocasiones la dopamina señaliza respuestas negativas, y estudios más recientes asocian su liberación a un \enquote{error en la predicción}.

Además, la activación de neuronas dopaminérgicas es solo uno de los componentes de todos los procesos neuronales que ocurren durante la emoción de recompensa. Ratas carentes de dopamina aún exhiben comportamientos hedónicos ante el azúcar y la cocaína, lo que indica que este neurotransmisor no es el único factor relevante.

Peor aún, el patrón de comportamiento de realizar algo repetidamente incluso en detrimento propio puede aplicarse a conductas como el juego, la comida o el sexo, áreas complicadísimas de estudiar, pues es imposible construir un modelo humano a partir de una rata con adicción a ir de compras.

Peor aún, el patrón de comportamiento de realizar algo repetidamente incluso en detrimento propio puede aplicarse a conductas como el juego, la comida o el sexo, áreas complicadísimas de estudiar, pues es imposible construir un modelo humano a partir de una rata con adicción a ir de compras. Hasta en roedores es difícil hallar patrones deterministas: grupos de ratones no desarrollan adicción a la cocaína en las mismas proporciones estando totalmente aislados que dentro de \enquote{ambientes enriquecidos} dotados de compañeros, tubos de polietileno y pelotas.

¿Qué paralelismos se pueden trazar entre el comportamiento de una rata y el de un humano? ¿Cómo se manifiesta la adicción a escala neuronal y cómo se relaciona realmente a las drogas? Nos topamos con el mantra repetido durante toda investigación neurocientifica: no lo sabemos. Desprovistos de afirmaciones universales, nos vemos obligados a analizar los efectos concretos del LSD.

\subsection{Farmacología del LSD.}

La dietilamida del ácido lisérgico es una droga semisintética derivada del ácido lisérgico extraído del ergot. Posee cuatro estereoisómeros, es decir, cuatro sustancias hermanas estructuralmente: l- y d- LSD y l- y d- iso-LSD. Solo el d-LSD es psicoactivo, y es al que nos referimos al decir \enquote{LSD}. Es hidrosoluble y se derrite a 83°C, y debido a su fragilidad ante la luz, la temperatura y la humedad suele ser conservado en su forma de sal de tartrato. Debido a su alta biodisponibilidad, la vía de administración más común es la oral. Originalmente se realizaba a través de ampollas, pero se popularizó el consumo a través de papeles secantes o terrones de azucar bañados y colocados sobre la lengua. Distintas dosis de LSD configuran distintos efectos.

En un individuo de entre 50 y 70 kg de peso, la mínima dosis reconocible es de 25 $\mu$g y produce algunas alteraciones cognitivas. La dosis estándar está entre 70 y 100 $\mu$g, y produce una acción de 6 a 10 horas, ya con efectos visionarios. A partir de 300 $\mu$g comienzan las dosis altas, con efectos intensos durante más de 10 horas.

La dosis letal media en distintos animales está entre 0.3 y 16.5 mg / kg, siendo 1 mg / kg en monos. No se ha encontrado la dosis letal en humanos, pues no se conocen casos de muertes por LSD. En una ocasión ocho individuos confundieron LSD con cocaina y consumieron una dosis altisima, hallando 1-7 mg de la sustancia por cada 100 mL de su sangre. Sufrieron de estados comatosos, hipertermia, vómitos, sangrados gástricos y problemas respiratorios, pero con tratamiento hospitalario, todos sobrevivieron sin secuelas. Es preciso afirmar entonces que el margen de seguridad del LSD es extraordinariamente alto.

El LSD presenta una tolerancia elevadísima. Después de dosis de entre 5 y 100 microgramos administradas durante 3 a 6 días, los voluntarios desarrollan una fuerte tolerancia. Esta desaparece en torno a los 4 días sin uso. Más allá de la tolerancia, existe un consenso general en que el LSD no es adictivo, es decir, no produce los comportamientos de búsqueda compulsiva de más dosis.

Se observa que esta sustancia tiene unas propiedades excepcionales, sin embargo que sea farmacológicamente segura no implica que su uso en general sea seguro. Existen casos documentados de autolesión y suicidio bajo los efectos de esta droga, si bien el número es reducido. Más detalles al respecto se dan en el capítulo sobre su historia contemporánea.

\subsection{Efectos del LSD.}

Una dosis estándar de LSD altera de manera significativa el estado de consciencia, con una tendencia a la euforia, es decir, un intenso estado de felicidad y bienestar; un impulso en la capacidad de introspección y la estimulación de procesos Freudianos primarios, es decir, los deseos instintivos se desatan, inhibiendo la influencia del ego y la sociedad sobre el individuo. El efecto más característico son las alteraciones en la percepción como ilusiones, visiones, pseudoalucinaciones, sinestesias - la capacidad de cruzar sentidos - y alteraciones en el pensamiento y la percepción temporal. Mención especial merecen los cambios que provoca sobre la percepción de uno mismo y la función del ego. El LSD (especialmente en dosis altas) debilita la frontera que separa al \enquote{yo} del resto del mundo, un fenómeno conocido como \enquote{disolución} o \enquote{muerte del ego}.

La potencia inigualada del LSD es un arma de doble filo, pues al igual que es capaz de proporcionar una buena experiencia y efectos psicológicos positivos a largo plazo, también puede provocar experiencias traumáticas (llamadas coloquialmente \enquote{malos viajes}) con efectos negativos como cambios de humor y, a veces, escenas retrospectivas que pueden resultar nocivas. Es difícil estudiar los efectos del LSD en el pensamiento, pues no podemos introducirnos en la mente de nadie y la dosis estándar ya es suficiente para dificultar la comunicación durante los experimentos. Se puede decir que el efecto del LSD reduce la destreza en pruebas de atención y concentración, habilidades psicomotoras y aritméricas, memoria visual y noción temporal – se tiende a sobreestimar los intervalos temporales. Los procesos de aprendizaje no se ven afectados. Ciertas interpretaciones argumentan que las funciones intelectuales sufren un regreso a un estado más temprano de desarrollo. No se conoce ninguna facultad perjudicada de manera crónica por el uso de LSD. En muchas personas produce mareos y agitación interna.

\subsection{Mecanismo de acción: el sistema serotoninérgico.}

La serotonina (también denominada 5-hidroxitriptamina o 5-HT) es un neurotransmisor que se produce a partir del triptófano en pocas neuronas (contadas en millares). Estas neuronas se ubican principalmente en los núcleos de rafé del mesencéfalo, y proyectan sus conexiones a regiones como el hipocampo, la corteza cerebral, el NAc y núcleos dopaminérgicos como la VTA, entre otras (Figura 9). Es particularmente importante la conexión de los núcleos de rafé con el locus coeruleus (LC), pues esta región se encarga de la liberación de noradrenalina y tiene conexiones con el cerebelo, el tálamo, el hipotálamo, la corteza y el hipocampo.

Es evidente que, a pesar de ser reducidas en número, las abundantes conexiones salientes de las neuronas serotoninérgicas (hasta 500.000 por neurona) hacen de la 5-HT un neurotransmisor crucial en procesos tan diferentes como la regulación del ánimo, la temperatura corporal y hasta el tracto gastrointestinal. La deficiencia en niveles de serotonina se asocia a trastornos depresivos, que son tratados con varias formas de psicoterapia y fármacos antidepresivos (normalmente inhibidores selectivos de recaptación de serotonina o SSRIs, como la fluoxetina y la sertralina).

El sistema serotoninérgico está también relacionado con el filtro de información en el cerebro. No todos los estimulos del medio son igual de importantes, y algunos - como el sonido del oleaje - son continuos y repetitivos, por lo que no merecen atención constante. El cerebro tiene una capacidad de procesamiento limitada, por lo que automáticamente descarta esta información para dejar espacio a otra más valiosa, evitando además una sobrecarga sensorial. Una alteración en este sistema de cribado podría explicar los efectos estimulantes y alucinógenos del LSD.

La estructura molecular del LSD es similar a la de la serotonina, lo que le permite unirse a algunos de sus receptores, aunque con menos fuerza que la propia serotonina (se dice que es un agonista parcial). Esta interacción altera el comportamiento del sistema serotoninérgico, aunque no se conoce plenamente cómo. Las neuronas del sistema serotoninérgico cuentan con 14 receptores de serotonina distintos (y se teoriza un 15º), siendo algunos inhibitorios y otros excitatorios, y la gran mayoría metabotrópicos (es decir, que no controlan directamente la apertura de un canal, sino que modulan la actividad de la neurona por medio de segundos mensajeros). Estos receptores se nombran como \enquote{5-HT$_{XY}$} (receptor de serotonina Y de la clase X).

Concretamente actúa como agonista parcial en los receptoers 5-HT$_{1A}$, 5-HT$_{2A}$ y 5-HT$_{2C}$, pero también sobre receptores de otros neurotransmisores como el dopaminérgico D$_2$ y el adrenérgico $\alpha_2$. Un efecto secundario de estimular los receptores 5-HT$_{2A}$ es la activación de transmisión glutamatérgica en la corteza frontal. Este es un patrón clave compartido por el LSD y otros alucinógenos serotoninérgicos, y se cree que esta activación podría causar una perturbación en los sistemas de filtro de procesos sensoriales y cognitivos. La tolerancia podría ocurrir debido a una reducción en la densidad de receptores 5-HT$_{2A}$.

% Figura

\section{Historia contemporánea y usos.}

Menos de una hora después de que Hofmann se administrase la dosis, estaba claro que el LSD era el culpable de su anómalo estado. Ayudado por su asistente, regresó en bicicleta a casa atravesando un paisaje ondulante. Tumbado en el sofá, su alrededor se tornaba grotesco. Su vecina que traía leche se convirtió en una malvada bruja. Estaba aterrorizado de que su experimento dejase a una familia sin padre, mas cuando llegó el médico, este no aconsejó más que dejar pasar el tiempo, pues sus constantes vitales eran correctas. Gran acierto, ya que de aquél escenario macabro descendió a la realidad, pudiendo disfrutar del residuo de sinestesias e imágenes caleidoscópicas. Al día siguiente se encontraba con la cabeza totalmente fresca, pudiendo recordar toda su experiencia.

Hofmann visionó en el LSD una droga de enormísima utilidad en farmacología, neurología y particularmente en psiquiatría. Jamás habría podido anticipar sin embargo que fuese a ser utilizada de manera recreativa después de lo vivido.

Tras informar a su supervisor, el departamento farmacológico de Sandoz se puso manos a la obra a estudiar la tolerancia y toxicidad de la sustancia. Los primeros exámenes en animales mostraban a gatos que observaban a la nada y en lugar de atacar a ratones, los miraban horrorizados. Las arañas construían telas más eficientes en dosis bajas pero rudimentarias en dosis altas. Al introducirse en una comunidad de chimpancés, incluso si solo unos pocos individuos lo tomaban, el grupo entero entraba en conflicto debido a que estos no seguían las leyes jerárquicas de su tribu.

Demostrada la seguridad del LSD y probado sistemáticamente en humanos, Sandoz lo comercializó como un fármaco experimental: el Delysid. Entre sus usos se citaba además la autoexperimentación del psiquiatra para ganar perspectiva sobre la emoción interna de sus pacientes, pues la experiencia servía como modelo de distintas afecciones.

\subsection{Terapia psicodélica, arte y recreación, prohibición.}

El LSD no actúa como un medicamento \textit{per se}, es decir, no arregla - que sepamos - ningún desequilibrio químico como sí lo hace un antidepresivo. Sin embargo, dados los peculiares efectos que produce sobre la consciencia, sus usos como fármaco auxiliar en psicoterapia y psicoanálisis han sido muy abundantes. Al llevar al extremo el estado interno del usuario y hacer que reaparezcan recuerdos olvidados (en forma de \enquote{reviviscencia}), el LSD es una herramienta útil para liberar material reprimido, haciéndolo eficiente para tratar el trauma. Además, al disolver la barrera entre el tú y el yo, facilita el desprendimiento de problemas centrados en el ego. También se ha utilizado en pacientes terminales de cáncer que desarrollan mucha tolerancia a los analgésicos, aunque esto plantea cuestiones éticas. La experiencia de disociación corporal impide que el dolor penetre en la consciencia y facilita el coraje para afrontar la propia muerte. Este uso requiere supervisión y una preparación especial, a menudo siendo útil que un psicoterapeuta o una figura religiosa guíe la experiencia. El célebre escritor Aldous Huxley pidió en su lecho de muerte a su mujer ser inyectado con 100 microgramos, varias dosis si fuese necesario.

A principios de los 60 el LSD se extiende como la pólvora por occidente, con una influencia particular sobre la sociedad estadounidense que se ve acompañada casual o causalmente por el nacimiento del movimiento hippie. El doctor Timothy Leary de la universidad de Harvard fue una figura muy relevante en la difusión de LSD en los Estados Unidos. Realizó exámenes sobre su uso en miembros del clero, convictos y artistas, aunque sus obras rápidamente perdieron el carácter cientifico y fue expulsado de la universidad. Convertido al hinduísmo, pasó a ser uno de los principales líderes del movimiento hippie. Con el grito estampado hasta la saciedad de \enquote{Turn on, tune in, drop out}, alentaba a la juventud a consumir LSD (turn on) para explorar su mundo interior (tune in) y finalmente desprenderse de la vida burguesa, los estudios, el trabajo y toda cadena con el cántico \enquote{drop out} (Figura \ref{albums}).

\begin{figure}[H]
	\centering
	\includegraphics[height=.2\textheight]{media/10-revolver.jpg}
	\includegraphics[height=.2\textheight]{media/10-leary.jpg}
	\includegraphics[height=.2\textheight]{media/10-dsotm.png}
	\caption{Carátulas de los álbumes \textit{Revolver} (The Beatles), \textit{Turn On, Tune In, Drop Out} (Timothy Leary) y \textit{The Dark Side of the Moon} (Pink Floyd), muy influídos por el LSD.}
	\label{albums}
\end{figure}

Esta subversión de las estructuras convencionales era un pronunciamiento abierto que desafiaba - recordando a los chimpancés que perdieron su estructura jerárquica - a toda autoridad social y política. El Dr. Leary fue arrestado en Kabul y encarcelado por posesión de drogas. En 1976, ya libre, se mantuvo ocupado con cuestiones como la relación entre el SNC y el espacio interestelar.

El LSD siguió una ruta histórica similar a la de la mescalina: comenzó como un químico de interés en agrupaciones cientificas y psiquiátricas y se inmiscuyó entre intelectuales. Las experiencias estéticas e introspectivas inducidas lideraban el proceso creativo de los artistas, y de sus numerosas obras nació el arte psicodélico, que comprende toda creación realizada tras el uso de LSD y otros psicotrópicos como la mescalina o la psilocibina. El libro \textit{Psychedelic art} contiene una recopilación excelente de este género.

En la música grupos tan populares como los Beatles fueron marcados por esta sustancia. Discos como \textit{Revolver}, o \textit{Sgt. Pepper’s Lonely Hearts Club Band} y canciones como \textit{Lucy in the Sky with Diamonds} (título que hoy sirve de alias a esta droga) recibieron una influencia estética propia del LSD. En conjunto, bandas coetáneas como Pink Floyd, Jefferson Airplane y The Grateful Dead concibieron el género del \textit{rock psicodélico}.

El LSD se introdujo también en las clases populares. Hofmann había previsto curiosidad por parte de los artistas e intelectuales, pero jamás pensó que su creación fuese a ser empleada como un embriagador común, y este uso provocó varios incidentes. Los experimentos clínicos y universitarios con la droga pasaron de ser compartidos en publicaciones cientificas a ser presentados delante de todo el mundo en revistas y periódicos, donde las conclusiones eran exageradas. Los reportajes acerca de esta droga ya no se daban en tercera persona, sino que los mismos periodistas la consumían y relataban posteriormente sus experiencias. En los mercados estadounidenses aparecían libros relatando los efectos del LSD, muchas veces exaltándolos.

Toda esta literatura implantó en la cultura popular una idea falsa: que el solo uso de esta medicina era suficiente para lograr efectos milagrosos. Bajo semejante concepción, comenzó el imperio de la autoexperimentación. Los sesenta, con una crisis existencial de la sociedad estadounidense, la completa legalidad de la sustancia y la expiración de las patentes de Sandoz hicieron al LSD omnipresente. Como era de esperar, las experiencias populares se parecían más a las primeras de Albert Hofmann. \enquote{Malos viajes}, desorientación y pánico eran el producto habitual del experimento propio, a veces provocando accidentes y crímenes.

Entre 1964 y 1966 la polémica del LSD reinaba, con entusiastas diciendo que era una sustancia mágica y otros que señalaban los accidentes y crímenes que se cometian bajo sus efectos. Sandoz sufrió demandas masivas acerca de las propiedades de la sustancia. Finalmente en agosto de 1965 dejaron de producir y exportar públicamente el Delysid. A cambio, ofrecieron entregarlo a investigadores cualificados de todo el mundo, con asistencia tanto técnica como en muchos casos financiera. Sumado a una detallada descripción en el \enquote{Catalogue of Literature on Delysid}, con la que se inhibió en buena medida el uso indebido.

Sandoz sin embargo no podía controlar todas las posibles excepciones. Las ideas erróneas en circulación y la total ausencia de legislación forzaron a Sandoz a cesar la producción de LSD y de alucinógenos similares como la psilocibina. A continuación siguió el Convenio sobre Sustancias Psicotrópicas de la ONU y el establecimiento de estatutos como el Controlled Substances Act en Estados Unidos, que no solo prohibía su posesión y distribución, sino que descartaba cualquier uso terapéutico de la sustancia. Su asociación a una \enquote{droga de la locura} disuadió a los psiquiatras de continuar empleándolo. El LSD, como condenado por una trágica maldición familiar, fue llevado al ostracismo al igual que su padre el ergot tres siglos antes.

El declive en uso de LSD ha afectado notablemente a su producción. En estudios psiquiátricos y neurobiológicos se ha visto sustituído por sustancias como la psilocibina encontrada en las setas alucinógenas, no solo por su semejanza en farmacología y efectos, sino porque su tiempo de acción más breve facilita su estudio. Como droga ilegal también se ha visto desplazado por derivados del cáñamo como el hachís y drogas sintéticas como la heroína y las anfetaminas, siendo habitualmente más tóxicos los sustitutos.

\subsection{Riesgos asociados al LSD.}

Actualmente la legislación no atribuye usos terapéuticos al LSD. Existen opiniones contrarias que argumentan que no hay peligro en el uso en entornos profesionales, algunas llegando a afirmar que el existente fuera de estos está relacionado con la clandestinidad y no con la sustancia. Es innegable que el uso de LSD está asociado a un conjunto de riesgos que deben ser conocidos tanto si se hace en un entorno profesional con supervisión médica como si no.

La desorientación intrínseca a todo experimento con LSD hace imposible descartar episodios de crisis, por más que se pueda minimizar mediante preparación. En los peores casos, la psicosis inhibirá la percepción de riesgo del individuo, produciendo accidentes fatales. En su defecto, una experiencia con visiones mortiferas puede conducir al suicidio. Aun si no son casos tan comunes, han de servir de advertencia. Frank Olson, un doctor estadounidense, consumió altas dosis de LSD sin su conocimiento. Se suicidó saltando por una ventana. Había sido víctima de experimentos farmacológicos del ejército estadounidense, y no fue hasta pasados 15 años que, con la revelación del proyecto MK Ultra, el director de la CIA William Colby y el presidente del gobierno Gerald Ford presentaron sus disculpas a la familia.

Es importante también analizar si el LSD es el fármaco óptimo para el bienestar del paciente, suministrándolo solo en los casos adecuados. Como el LSD intensifica el estado mental del momento, ofrecérselo a un paciente con la intención de curar un mal ánimo puede ser muy perjudicial. Tampoco debe ser utilizado sobre personalidades inestables, como aquéllas con tendencia a la psicosis. Esto incluye por lo general a la gente joven.

Es necesario un comentario acerca del uso personal que recibe el LSD. Existe un problema intrínseco al uso personal de cualquier droga. Si por definición estas actúan sobre el sistema nervioso afectando a la emoción y la percepción - y nuestro juicio se basa precisamente en estas facultades - entonces nuestra toma de decisiones se verá comprometida, incluyendo la propia decisión de consumir la droga. Ideas aceptadas o rechazadas en un estado de conciencia pueden no serlo en otro incluso a pesar de realizarse una planificación previa. Esta cuestión plantea un debate acerca de la relación del libre albedrío con el sistema nervioso, pero no lo trataremos.

Es imposible realizar un estudio cientifico acerca del uso personal de LSD, ya que de establecer un entorno controlado y realizar un seguimiento de cada individuo, el uso dejaría de ser personal. Dependemos entonces de la \enquote{sabiduría popular} relatada por periodistas y particulares como el propio Albert Hofmann. En diversas fuentes se habla por ejemplo del concepto de \enquote{set y setting} como determinante para que los sentimientos durante una experiencia sean predominantemente positivos. Se denomina \enquote{set} al conjunto de factores internos como el estado de ánimo previo y las expectativas, y \enquote{setting} a los factores externos como el carácter del entorno, su iluminación, su ruido ambiente o las personas presentes. También se menciona la importancia de una persona de confianza como sustento emocional capaz de solicitar asistencia médica en caso de ser necesario. Se hace especial énfasis en la dosis, recomendándose empezar por unas muy bajas y llevar un registro con los efectos de cada una.

Finalmente es menester recordar que el LSD es ilegal, y que en consecuencia el mercado negro es la única vía de obtención. Esto plantea riesgos totalmente ajenos a los farmacológicos. Al no estar regulados, la pureza de los productos clandestinos es desconocida, y es común encontrarlos contaminados por otras sustancias. Además del problema de la pureza está el de la dosis: incluso si se adjunta la medida en microgramos (también llamados \enquote{\textit{gammas}}), esta puede estar alterada. Dada la abismal diferencia entre dosis de 25, 100 y 600 microgramos de LSD, la incertidumbre al momento de consumir un producto clandestino hace mucho más comunes las malas experiencias.

\newpage

\section{Conclusion.}

Even after this study I find myself filled with contradictions and I'm incapable of offering a consistent opinion on the moral matters. Is psychedelic therapy a potential replacement to conventional therapy or just a complement? Being able to use a substance without restrictions is freedom or slavery for users? Albert Hofmann himself ended up calling his creation a \enquote{problem child}. As with any big discovery --- the printing press, the steam engine, the Internet --- society needs some time to adapt.

Having read its historical background, removing the prohibition of its therapeutic use could make up for a very interesting field of research. Understanding serotonin's role in psychic and metabolic proceses and constructing an evolutionary map on its procedence could assist on the comprehension of brain circuits as functional components of conscious experience.

It's safe to state that any attempt to reintegrate LSD as a pharmaceutical in society won't happen without authorization from the law. Because it's such an unstable compound, harmed both by air and light, lab material and very broad knowledge on chemistry are mandatory for its synthesis and manipulation. Not the same can be said about other hallucinogens. Because they come from vegetal and fungal sources, mescaline, DMT or psylocibin are much more accessible. It's then practically impossible to completely prohibit hallucinogens (and in truth, any drug) due to the decentralized nature of their manufacture.

After this study it's possible to establish a series of statements whose truth I consider non-negotiable:

\begin{itemize}
	\item The complete elimination of a substance that comes from a fungus or plan that is easy enough to grow is impossible (or at least very expensive).
	\item Some drugs are used or have been used with therapeutic ends.
	\item Some drugs are used or have been used with recreational ends.
	\item A drug's usage implies personal and social risks.
	\item If this usage is done with the necessary knowledge, the risk can be fatal.
\end{itemize}

Which conclusions should be drawn from this is a deep topic for debate which involves psychiatrists and psychologists, but also hystorians, philosophers and governments. Two main discussions exist. First, the one that asks whether the use of a certain drug is problematic. For example, many civilisations through history have used opium and laudanum as sources of analgesia and entertainment, and now they are illegal. After this debate, the second question comes into place: the one referring to prohibition. In some places like the Netherlands it's considered that the use of these substances is something negative, but even then they have eliminated prohibition. The official reason is that this \enquote{prevents people who use soft drugs from coming into contact with hard drugs}, a certainly pragmatic way to handle the situation.

The disagreement is far from closed, however, knowing that some drugs will be perpetually in our society --- and that surrounding them with misticism will only increase the number of accidents --- the need for information on their use and risks is needed.


% Bibliografía

\nocite{*}
\bibliographystyle{IEEEtran}
\bibliography{refs}

\end{document}
