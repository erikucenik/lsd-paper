\section{Introduction.}

Rye crops display purple spots on their flowers during the winter. These spots, initially thought to be an excess of sap, encapsulate the monster responsible for two of medieval Europe's worst endemic diseases. They harbor such a variety of chemicals that, when not causing spasms, they serve as medicine, and in a bizarre twist of fate, they also gave us the psychedelic drug that defined the 20th century, from the Vietnam War to the hippie pacifist movement.

Lysergic acid's diethylamide, or LSD, born from this havoc, is a molecule with miraculous properties, both pharmacological and psychiatric. With immense potency at doses of mere micrograms, it has played a starring role in recent social movements. What therapeutic potential does this substance hold? What were the consequences of its discovery? We present a study on its nature and history, its effects on individuals along with the associated risks, and, above all, the biological foundations that explain its impact on consciousness.

\newpage
