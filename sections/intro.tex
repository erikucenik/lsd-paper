\section{Introduction.}

In winter harvests, purple-colored stains show up in rye flowers. These stains, initially thought to be excess sap, encapsulate the monster responsible for two of medieval Europe's endemic diseases. They hold chemical compounds so varied that when they don't cause spasms, they serve as medicine, and in a bizarre coincidence they gave us the psychedelic drug that marked the 20th century, from the Vietnam War to hippie pacifism.

Lysergic acid's diethylamide or LSD, born from this havoc, is a molecule with miraculous properties both pharmacologically and psychiatrically. With a huge potenci in microgram dosages, it has starred the most recent social movements. What therapeutic potential does this substance have? What were the consequences of its discovery? We present a study on its natur and history, the effects it causes on the individual and the associated risks, and over all, the biological fundamentals which explain its action upon conscience.

\newpage
