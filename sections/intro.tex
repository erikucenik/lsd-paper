\section{Introducción.}

Las cosechas de centeno presentan en invierno manchas de color púrpura en sus flores. Estas manchas, inicialmente consideradas exceso de savia, encapsulan al monstruo responsable de dos enfermedades endémicas de la Europa medieval. Albergan sustancias químicas tan variadas que cuando no causan espasmos sirven como medicina, y en una bizarra casualidad nos dieron la droga psicodélica que marcó el siglo XX, desde la guerra de Vietnam hasta el pacifismo hippie.

La dietilamida del ácido lisérgico o LSD, hija de estos estragos, es una molécula de propiedades milagrosas tanto farmacológica como psiquiátricamente. Con una potencia descomunal en dosis de tan solo microgramos, ha protagonizado los movimientos sociales más recientes. ¿Qué potencial terapéutico tiene esta sustancia? ¿Cuáles fueron las consecuencias de su descubrimiento? En este documento planteamos un estudio sobre su naturaleza e historia, los efectos que causa en el individuo con los riesgos asociados y ante todo los fundamentos biológicos que explican su acción en la consciencia.
