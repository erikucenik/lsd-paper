\section{Conclusion.}

Even after this study I find myself filled with contradictions and I'm incapable of offering a consistent opinion on the moral matters. Is psychedelic therapy a potential replacement to conventional therapy or just a complement? Being able to use a substance without restrictions is freedom or slavery for users? Albert Hofmann himself ended up calling his creation a \enquote{problem child}. As with any big discovery --- the printing press, the steam engine, the Internet --- society needs some time to adapt.

Having read its historical background, removing the prohibition of its therapeutic use could make up for a very interesting field of research. Understanding serotonin's role in psychic and metabolic proceses and constructing an evolutionary map on its procedence could assist on the comprehension of brain circuits as functional components of conscious experience.

It's safe to state that any attempt to reintegrate LSD as a pharmaceutical in society won't happen without authorization from the law. Because it's such an unstable compound, harmed both by air and light, lab material and very broad knowledge on chemistry are mandatory for its synthesis and manipulation. Not the same can be said about other hallucinogens. Because they come from vegetal and fungal sources, mescaline, DMT or psylocibin are much more accessible. It's then practically impossible to completely prohibit hallucinogens (and in truth, any drug) due to the decentralized nature of their manufacture.

After this study it's possible to establish a series of statements whose truth I consider non-negotiable:

\begin{itemize}
	\item The complete elimination of a substance that comes from a fungus or plan that is easy enough to grow is impossible (or at least very expensive).
	\item Some drugs are used or have been used with therapeutic ends.
	\item Some drugs are used or have been used with recreational ends.
	\item A drug's usage implies personal and social risks.
	\item If this usage is done with the necessary knowledge, the risk can be fatal.
\end{itemize}

Which conclusions should be drawn from this is a deep topic for debate which involves psychiatrists and psychologists, but also hystorians, philosophers and governments. Two main discussions exist. First, the one that asks whether the use of a certain drug is problematic. For example, many civilisations through history have used opium and laudanum as sources of analgesia and entertainment, and now they are illegal. After this debate, the second question comes into place: the one referring to prohibition. In some places like the Netherlands it's considered that the use of these substances is something negative, but even then they have eliminated prohibition. The official reason is that this \enquote{prevents people who use soft drugs from coming into contact with hard drugs}, a certainly pragmatic way to handle the situation.

The disagreement is far from closed, however, knowing that some drugs will be perpetually in our society --- and that surrounding them with misticism will only increase the number of accidents --- the need for information on their use and risks is needed.
