\section{Conclusión.}

Incluso tras este estudio me hallo lleno de contradicciones y soy incapaz de ofrecer una opinión consistente sobre las cuestiones morales. ¿Es la terapia psicodélica un potencial sustituto a las terapias convencionales o un complemento? ¿Poder hacer uso de la sustancia sin restricciones es para los usuarios libertad o esclavitud? El propio Albert Hofmann terminó denominando a su creación como \enquote{un hijo problemático}. Como ocurre con cualquier gran descubrimiento - la imprenta, la máquina de vapor, Internet - la sociedad necesita un tiempo de margen para adaptarse.

Habiendo leído sus antecedentes históricos, derogar la prohibición del uso terapéutico de LSD constituiría una rama de investigación interesante. Entender el rol de la serotonina en los procesos psíquicos y metabólicos y realizar un mapa evolutivo de su procedencia puede asistir en la comprensión de los circuitos cerebrales como componentes funcionales de la experiencia consciente.

Es probable afirmar que cualquier intento de reintegración del LSD como fármaco en la sociedad no sucederá sin autorización por parte de la ley. Al ser una sustancia sumamente inestable, sensible tanto al aire como a la luz, son obligatorios conocimientos extensos en química y material de laboratorio especial para su síntesis y manejo. No se puede decir lo mismo de otros alucinógenos. Al provenir de fuentes vegetales o fúngicas, la mescalina, la DMT o la psilocibina son mucho más accesibles. Es entonces prácticamente imposible la prohibición total de los alucinógenos (y en realidad otras drogas) por la naturaleza descentralizada de su manufactura.

Tras el estudio aquí realizado es posible establecer una serie de enunciados cuya verdad considero innegable:

\begin{itemize}
	\item La eliminación total de una sustancia proveniente de un hongo o planta suficientemente fácil de cultivar es imposible (o al menos muy costosa).
	\item Algunas drogas se usan o han sido usadas con fines terapéuticos.
	\item Algunas drogas se usan o han sido usadas con fines recreativos.
	\item El uso de una droga conlleva riesgos personales y sociales.
	\item Si este uso se hace sin el conocimiento necesario, el riesgo puede ser fatal.
\end{itemize}

Qué conclusiones derivar es un tema de debate profundo que involucra a psiquiátras y psicólogos, pero también a historiadores, filósofos y gobiernos. Existen dos discusiones principales al respecto. Primero, aquélla referente a si el uso de determinada droga es algo problemático. Por ejemplo, muchas civilizaciones a lo largo de la historia han utilizado el opio y el láudano como fuentes de analgesia y divertimento, y ahora ambos son ilegales. Habiéndose dado este debate ocurre la segunda discusión: aquélla referente a la prohibición. En algunos lugares como los Países Bajos se considera que el uso de sustancias es algo negativo, pero aun así se ha derogado la prohibición, siendo el motivo oficial que esto \enquote{previene que las personas que consumen drogas blandas entren en contacto con drogas duras}, un manejo de la situación ciertamente pragmático.

El desacuerdo está lejos de cerrarse, sin embargo sabiendo que algunas drogas estarán de manera perpetua en la sociedad – y que rodearlas de misticismo solo aumentará el número de accidentes – se hace evidente la necesidad de información sobre su uso y riesgos.
